%%%%%%%%%%%%%%%%%%%%%%%%%%%%%%%%%%%%%%%%%%%%%%%%%%%%%%%%%%%%%%%
\documentclass[11pt]{beamer}
\usetheme{Berlin}
\date{20 июня 2024}
\usepackage{tikz}
\usepackage{colortbl}
\usepackage[russian]{babel}

% define new colors
\definecolor{Blue}{RGB}{73,73,256}
\definecolor{LightBlue}{RGB}{202,201,229}
\definecolor{VeryLightBlue}{RGB}{230,230,242}

%Information to be included in the title page:
\title{Электронные машины для голосования по сравнению с традиционными методами (handout)}
\author{выполнил: Дубровин И.}

\begin{document}

\frame{\titlepage}

\begin{frame}
\frametitle{Оглавление}
\tableofcontents
\end{frame}

\section{Введение и цели}

\begin{frame}
\frametitle{Введение и цели}
    \begin{column}[h]{0.5\linewidth}  
        \begin{itemize}
        \item Кажется, что методы электронного голосования в скорем \alert{заменят} традиционные.\newline
        \item Тогда электронные системы должны соответсвовать определённым критериям \alert{удобства} и \alert{эффективности}.
        \end{itemize}
    \end{column}
    \begin{column}[h]{0.5\linewidth}
        \begin{itemize}
        \item Выяснить удобность методов электронного голосования относительно других. \newline
        
        \item Установить эффективность метода (защиту от ошибок).
        \end{itemize}
    \end{column}
\end{frame}

\section{Материалы и методы}

\begin{frame}
\frametitle{Материалы и методы.}

    \begin{itemize}
        \item \small Для объективной оценки необходимо проведение полностью вымышленного голосования.\newline
        
        \item \small Методы голосования: бумажные бюллетени, перфокарты и механические рычажные машины.\newline

        \item \small Для оценки субъективного восприятия пользователями каждого метода голосования был использован опросник SUS (System Usability Scale). \newline

        \item \small В качестве DRE (Direct Recording Electronic) использовалась авторская система \alert{VoteBox}.
    \end{itemize}
\end{frame}

\section{Эксперимент}

\begin{frame}
\frametitle{Эксперимент}
    \begin{itemize}
        \item Добровольцам предлагалось пройти процедуру голосования двумя способами: \alert{DRE} и один случайный другой метод.\newline
        \item Каждый раз происходило ознакомление с информацией о кандидатах и голосование после.\newline
        \item При выходе избирателя, происходили необходимые замеры.
    \end{itemize}
\end{frame}

\begin{frame}
\frametitle{Результаты}
\resizebox{11cm}{!}
{
    \begin{tabular}{|c|c|c|c|c|}
    \rowcolor{Blue}\hline
    {\color[HTML]{FFFFFF} Категория} & {\color[HTML]{FFFFFF} DRE} & {\color[HTML]{FFFFFF} Бумажная билютень} & {\color[HTML]{FFFFFF} Рычажковый} & {\color[HTML]{FFFFFF} Перфокарта} \\ \hline \rowcolor{LightBlue}
    \begin{tabular}[c]{@{}c@{}}Среднее время заполнения\\ (сек)\end{tabular} & 442.3 & 255.7 & 241.6  & 239.1 \\ \hline \rowcolor{VeryLightBlue}
    \begin{tabular}[c]{@{}c@{}}Средняя удовлетворённость\\ SUS (\%)\end{tabular} & 86.1  & 81.3 & 71.5 & 69.0 \\ \hline \rowcolor{LightBlue}
    \begin{tabular}[c]{@{}c@{}}Ошибки дополнительного\\ голоса (\%)\end{tabular} & .000  & .000 & .000 & .000 \\ \hline \rowcolor{VeryLightBlue}
    Ошибки неучёта голоса (\%)                                                   & .002  & .002 & .006 & .000 \\ \hline \rowcolor{LightBlue}
    \begin{tabular}[c]{@{}c@{}}Ошибки неправильного\\ выбора (\%)\end{tabular}   & .120  & .002 & .011 & .002 \\ \hline \rowcolor{VeryLightBlue}
    Другие ошибки (\%)                                                           & .130  & .004 & .017 & .002 \\ \hline
    \end{tabular}
}
\end{frame}

\section{Обсуждение результатов и заключение}

\begin{frame}
\frametitle{Обсуждение результатов и заключение}
    \begin{block}{Высокая удовлетворенность}
        \alert{отсутствие явного преимущества DRE} по показателям эффективности, но избиратели отдают ей предпочтение.
    \end{block}

    \begin{alertblock}{Отсутстве значимого превосходства}
        В многочисленных исследованиях \alert{мы не увидели никаких сильных различий} между методами по эффективности.
    \end{alertblock}
    
    \begin{alertblock}{Возможность улучшения}
        VoteBox лишь прототип электронной системы. Однако возможности для улучшения в иследованиях с другими DRE \alert{не обнадеживают} с точки зрения количества ошибок. 
    \end{alertblock}
\end{frame}

\begin{frame}
    \centering
    \Huge{\Huge\calligra Спасибо за внимание!}
\end{frame}

\end{document}